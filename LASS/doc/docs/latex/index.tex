\hypertarget{index_intro}{}\section{Introduction}\label{intro}
LASS was written to provide musicians an environment for performing additive sound synthesis. It is unique from other systems in the way that it allows musicians to specify how 'loud' a sound should be heard. LASS then adjusts the sounds to the correct amplitude via a method called critical bands. The three main design goals for the project are expandability, ease of use, and efficiency. LASS is designed with a very modular architecture. No doubt, there will be features that need to be added - and future generations of students must be able to easily expand the system. The system was also designed to be easy for users. The interface to the classes were made as clear as possible and kept consistent across objects. Also, I've made extensive use of references instead of pointers to help ensure good memory management. Finally, LASS must also be efficient, for sound synthesis requires much calculation. This tool may not be quite as efficient as M4C, but tradeoffs in features and ease of use make it worth the while.\hypertarget{index_more}{}\section{More Documentation}\label{more}
for a brief tutorial on LASS, see tutorial.pdf 